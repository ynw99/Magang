\begin{center}
  \Large\textbf{KATA PENGANTAR}
\end{center}
\vspace{2ex}

\addcontentsline{toc}{chapter}{KATA PENGANTAR}

% Ubah paragraf-paragraf berikut sesuai dengan yang ingin diisi pada kata pengantar
Puji syukur kami panjatkan kepada Tuhan Yang Maha Esa karena hanya dengan rahmat dan hidayah-Nya penulis dapat melaksanakan magang/kerja praktik dan menyelesaikan laporan magang/kerja praktik di Google Bangkit dengan judul “MOBILE-BASED BATIK PATTERN RECOGNITION”.
Magang/kerja praktik telah dilaksanakan pada tanggal 15 Februari 2021 s.d. 25 Juni 2021.
Penulisan laporan magang/kerja praktik ini disusun sebagai syarat untuk memenuhi mata kuliah magang/kerja praktik di Departemen Teknik Komputer FTEIC-ITS Surabaya.
Dalam pelakasanaan maupun penulisan laporan magang/kerja praktik ini, penulis mengucapkan terima kasih atas bantuan, arahan, dan motivasi yang diberikan baik secara langsung ataupun tidak langsung.
Adapun pihak-pihak yang telah membantu dan membimbing kami dalam pelaksanaan kerja praktik yaitu:

\begin{enumerate}[nolistsep]

  \item Keluarga, Orang Tua, dan Saudara/i tercinta yang telah memberikan dorongan untuk mengikuti program ini.

  \item Bapak Dr. Supeno Mardi Susiki Nugroho, ST., MT. selaku Kepala Departemen Teknik Komputer, Fakultas Teknik Elektro dan Informatika Cerdas, Institut Teknologi Sepuluh Nopember.

  \item Bapak Reza Fuad Rachmadi, S.T., M.T., Ph.D dan Bapak Arief Kurniawan, ST., MT. selaku dosen pembimbing di Departemen Teknik Komputer FTEIC-ITS, yang telah memberikan arahan kepada kami selama mengikuti program Google Bangkit.

  \item Para pengurus program Google Bangkit, mentor-mentor yang telah membagikan ilmu dan masukan kepada kami, terutama Ms. Victoria Lestari yang menjadi mentor capstone project di Google Bangkit.

  \item Rekan-rekan program Google Bangkit yang juga memberikan ilmu lewat diskusi terutama yang menjadi anggota di dalam tim \textit{Bangkit Capstone Project}.

  \item Bapak-ibu dosen pengajar Departemen Teknik Komputer, atas pengajaran, bimbingan, serta perhatian yang diberikan kepada penulis selama ini.

\end{enumerate}

Penulis menyadari bahwa masih banyak kekurangan dalam perancangan dan pembuatan laporan magang/kerja praktik ini.
Besar harapan penulis untuk menerima saran dan kritik dari para pembaca.
Semoga buku laporan magang/kerja praktik ini dapat memberikan manfaaat bagi para pembaca, khususnya bagi penulis sendiri.

\begin{flushright}
  \begin{tabular}[b]{c}
    % Ubah kalimat berikut sesuai dengan tempat, bulan, dan tahun penulisan
    Surabaya, Juni 2021
    \\
    \\
    \\
    \\
    Penulis
  \end{tabular}
\end{flushright}
