% Ubah kalimat sesuai dengan judul dari bab ini
\chapter{PENDAHULUAN}

% Ubah konten-konten berikut sesuai dengan yang ingin diisi pada bab ini

\section{Latar Belakang}

Pesatnya perkembangan roket yang merupakan \lipsum[1][1-15]

\lipsum[2][1-10]

\section{Rumusan Permasalahan}

Masalah yang akan \lipsum[3][1-2] adalah:

\begin{enumerate}[nolistsep]

  \item Bagaimana cara \lipsum[3][3-5]

  \item \lipsum[3][6-8]

\end{enumerate}

\section{Tujuan}

Tujuan dari \lipsum[4][1-3] adalah:

\begin{enumerate}[nolistsep]

  \item Membuat \lipsum[4][4-5]

  \item \lipsum[4][6-9]

\end{enumerate}

\section{Manfaat}

Manfaat dari \lipsum[5][1-3] adalah:

\begin{enumerate}[nolistsep]

  \item Mempermudah \lipsum[5][4-5]

  \item \lipsum[5][6-10]

\end{enumerate}

\section{Waktu dan Tempat Pelaksanaan}

Kerja praktik akan dilaksanakan pada \lipsum[6][1-3]

\section{Metodologi Kerja Praktik}

Metode yang \lipsum[7][1-5] yaitu:

\begin{enumerate}[nolistsep]

  \item \textbf{Perumusan Masalah}

  Pada tahap ini \lipsum[7][6-9]

  \item \textbf{Studi Literatur}

  Pada tahap ini \lipsum[7][10-13]

  \item \textbf{Analisis dan Perancangan Sistem}

  Pada tahap ini \lipsum[8][1-2]

  \item \textbf{Implementasi Sistem}

  Pada tahap ini \lipsum[8][3-6]

  \item \textbf{Pengujian dan Evaluasi}

  Pada tahap ini \lipsum[8][7-12]

\end{enumerate}

\section{Sistematika Penulisan}

Laporan kerja praktik akan terbagi menjadi \lipsum[9][1] yaitu:

\begin{enumerate}[nolistsep]

  \item \textbf{Bab I Pendahuluan}

  Bab ini berisi \lipsum[9][2-4]

  \item \textbf{Bab II Profil Perusahaan}

  Bab ini berisi \lipsum[9][5-7]

  \item \textbf{Bab III Tinjauan Pustaka}

  Bab ini berisi \lipsum[9][8]

  \item \textbf{Bab IV Desain dan Implementasi}

  Bab ini berisi \lipsum[10][1-2]

  \item \textbf{Bab V Pengujian dan Evaluasi}

  Bab ini berisi \lipsum[10][3-4]

  \item \textbf{Bab VI Kesimpulan dan Saran}

  Bab ini berisi \lipsum[10][5-8]

\end{enumerate}
