% Ubah kalimat sesuai dengan judul dari bab ini
\chapter{PENDAHULUAN}

% Ubah konten-konten berikut sesuai dengan yang ingin diisi pada bab ini
Dalam Bab I dipaparkan penjalasan terkait latar belakang, tujuan, dan manfaat dari kerja praktik yang dilengkapi juga dengan sistematika dari penulisan laporan magang/kerja praktik ini.
\section{Latar Belakang}

Suatu bentuk kerjasama antara perguruan tinggi dan praktisi industri diperlukan sebagai bentuk implementasi kebijakan pemerintah yang ditujukan untuk peningkatan kualitas perguruan tinggi serta mendukung koordinasi di antara praktisi industri dan perguruan tinggi.

Bentuk kolaborasi yang dijalankan yaitu melakukan pekerjaan nyata atau kerja praktik di lingkungan perusahaan di mana keterampilan mahasiswa yang dipelajari di perguran tinggi dapat diterapkan.
Mata kuliah magang/kerja praktik mendukung mahasiswa untuk memahami lapangan atau tempat kerja.

Dalam program Google Bangkit, yang termasuk juga di Kampus Merdeka ini, memiliki program kerja \textit{Capstone Project} dimana para peserta Google Bangkit ditugaskan untuk membuat proyek yang bertemakan di \textbf{7 Tema dan Agenda Pembangunan dari RPJMN 2020--2024 dan 5 Area Prioritas dari Strategi Nasional Kecerdasan Artifisial} \cite{RPJMN} yang mana meliputi:

\begin{enumerate}[nolistsep]

  \item \textit{Economic Resilience}

  \item \textit{Competitive Human Resources}

  \item \textit{Infrastructure Development}

  \item \textit{National Identity \& Character Building}

  \item \textit{Political Stability, Rule of Law, National Security \& Public Services Transformation}

  \item \textit{Environmental Conservation, Disaster Resilience, and Climate Change}

  \item \textit{Regional Development}

\end{enumerate}

Dan juga ada 5 Area Prioritas Stranas KA \cite{stranas}:

\begin{enumerate}[nolistsep]

  \item \textit{Healthcare}

  \item \textit{Bureaucratic Reform}

  \item \textit{Education \& Research}

  \item \textit{Food Security}

  \item \textit{Mobility \& Smart City}

\end{enumerate}
Tujuan dari program ini adalah untuk menerapkan kemampuan penyelesaian masalah, melatih kerja sama dan kolaborasi, belajar cara mengatur projek, mengaplikasikan hasil pembelajaran ke pengaplikasian nyata, dan menyelesaiakan masalah nyata dengan teknologi.
\textit{Capstone Project} ini memerlukan untuk satu tim terdiri dari 3 program pembelajaran yang disediakan oleh Google Bangkit yaitu \textbf{Machine Learning, Mobile Development,} dan \textbf{Cloud Computing}.
\textit{Capstone Project} inilah yang menjadi bentuk magang/kerja praktik.

\section{Waktu dan Tempat Pelaksanaan}

Tempat dan waktu pelaksanaan magang/kerja praktik dilaksanakan secara daring mulai 3 Mei 2021 sampai dengan 15 Juni 2021.

\section{Tujuan}

Tujuan dari magang/kerja praktik dibagi menjadi dua, yaitu:

  \subsection{Tujuan Umum}
  Pada tujuan umum terdapat beberapa hal di antaranya adalah:
  
    \begin{enumerate}[nolistsep]

      \item Meningkatkan kesadaran akan dunia industri dan berkontribusi pada sistem pendidikan nasional.

      \item Membuka wawasan mahasiswa agar dapat mengetahui dan memahami aplikasi ilmunya di dunia industri.

      \item Sebagai wahana pembelajaran pengenalan pada lingkungan global kerja.

      \item Mahasiswa bisa mengetahui sistem kerja pada dunia industri sekaligus bisa mengadakan pendekatan kasus yang ada.

      \item Menumbuhkan dan menciptakan model berpikir konstruktif yang lebih berwawasan kepada siswa. Meningkatkan keterampilan praktis dalam bidang pengembangan \textit{machine learning}, \textit{mobile development}, \textit{cloud computing} dan mengaplikasikan secara langsung ilmu yang dipelajari dari Departemen Teknik Komputer Institut Teknologi Sepuluh Nopember.

    \end{enumerate}

  \subsection{Tujuan Khusus}
  Untuk tujuan khusus terdapat dua hal penting, yaitu:

    \begin{enumerate}[nolistsep]
      
      \item Untuk memenuhi beban satuan kredit semester (sks) yang harus ditempuh sebagai persyaratan akademis di Departemen Teknik Komputer.

      \item Mengembangkan pengetahuan, sikap, keterampilan dan kemampuan profesi melalui penerapan ilmu, latihan kerja dan pengamatan teknik.

      \item Meningkatkan wawasan mahasiswa di bidang \textit{Machine Learning} dan \textit{Cloud Computing}.

      \item Melaksanakan tugas dari program \textit{Capstone Project} dari Google Bangkit sebagai syarat kelulusan program Google Bangkit.

    \end{enumerate}

\section{Batasan Masalah}

Dalam penulisan laporan ini akan membahas tentang pengerjaan aplikasi \textbf{\textit{Mobile-Based Batik Pattern Recognition}} untuk tugas \textit{Capstone Project}.

\section{Metode Penulisan}

Metodologi yang digunakan pada penyusunan laporan magang/kerja praktik ini adalah:

\begin{enumerate}[nolistsep]

  \item \textbf{Studi Literatur}

  Penulis memanfaatkan referensi berupa data-data yang terdapat pada perpustakaan daring \cite{gultom2018batik}.

  \item \textbf{Metode Eksperimen}

  Penulis memperoleh data melalui eksperimen langsung pada objek guna mengamati peran dan hubungan masing-masing komponen objek, serta mencatat arti dan fungsi objek secara singkat dan jelas.

  \item \textbf{Metode Diskusi}

  Penulis mengumpulkan data dengan cara diskusi secara langsung kepada pembimbing atau mentor.

\end{enumerate}

\section{Sistematika Penulisan}

Berikut sistematika dari penulisan laporan ini:

\begin{enumerate}[nolistsep]

  \item \textbf{Bab I Pendahuluan}

  Pada BAB I dibahas mengenai latar belakang, waktu dan tempat pelaksanaan, tujuan, Batasan masalah, metodologi pengumpulan data serta sistematika penulisan.

  \item \textbf{Bab II Profil Perusahaan}

  Pada BAB II dibahas mengenai profil singkat dari Google Bangkit.

  \item \textbf{Bab III Tinjauan Pustaka}

  Pada BAB III dibahas mengenai pengertian metode-metode dalam pengerjaan \textit{Capstone Project} dan teori-teori penunjang yang berkaitan dengan kegiatan ini.

  \item \textbf{Bab IV Pembahasan}

  Pada BAB IV dibahas mengenai proses pembuatan aplikasi secara detail.

  \item \textbf{Bab V Penutup}

  Pada BAB V dibahas mengenai kesimpulan dan saran dari magang/kerja praktik yang sudah dilaksanakan.

\end{enumerate}
