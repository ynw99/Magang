% Ubah kalimat sesuai dengan judul dari bab ini
\chapter{KESIMPULAN DAN SARAN}

% Ubah konten-konten berikut sesuai dengan yang ingin diisi pada bab ini

\section{Kesimpulan}

Berdasarkan dari berjalannya proses pengembangan aplikasi \textbf{\textit{Mobile-Based Batik Recognition App}} ini, dapat disimpulkan beberapa hal:

  \begin{enumerate}[nolistsep]

    \item Aplikasi Mobile-based Batik Recognition App yang dinamai The Batik dapat bekerja.

    \item Dalam pengembangannya memerlukan 3 bidang keilmuan, yaitu \textit{Mobile Development}, \textit{Cloud Computing}, dan \textit{Machine Learning}.

    \item Untuk meringankan ukuran file aplikasi Android dari \textbf{The Batik}, tugas identifikasi batik dengan model \textit{Machine Learning} dilakukan di \textit{cloud} sehingga aplikasi Android tidak perlu menggunakan \textit{Machine Learning}.

  \end{enumerate}

\section{Saran}

Dari beberapa data yang diambil, terdapat beberapa saran untuk meningkatkan kualitas aplikasi ini:

  \begin{enumerate}[nolistsep]

    \item Menggunakan layanan \textit{\textbf{Cluster – Kubernetes}} di Google Cloud Platform untuk performa \textit{cloud computing} yang lebih tinggi.

    \item Menghadirkan fitur kamera untuk dapat memindai batik secara langsung.

    \item Meningkatkan akurasi model \textit{Machine Learning} dengan cara memperbanyak dataset dan melakukan \textit{hyperparameter tuning}.

  \end{enumerate}

% % Contoh input konten dari file terpisah
% \input{tabel/energi-kecepatan.tex}

% % Contoh penggunaan referensi dari tabel yang dibuat
% Sesuai dengan hasil pada Tabel \ref{tb:EnergiKecepatan}, didapatkan bahwa energi yang \lipsum[26]
